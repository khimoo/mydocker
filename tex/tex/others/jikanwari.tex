\listfiles
\documentclass{ltjsarticle}
\usepackage{tikz}

\begin{document}

% These set the width of a day and the height of an hour.
\newcommand*\daywidth{3.7cm}
\newcommand*\hourheight{2.3em}

% The entry style will have two options:
% * the first option sets how many hours the entry will be (i.e. its height);
% * the second option sets how many overlapping entries there are (thus
%   determining the width).


%\tikzset{entry/.style 2 args={
%    draw,
%    rectangle,
%    anchor=north west,
%    line width=0.4pt,
%    inner sep=0.3333em,
%    text width={\daywidth/#2-0.6666em-0.4pt},
%    minimum height=#1*\hourheight,
%    align=center
%}}
\tikzset{entry/.style 2 args={
    xshift=(0.5334em+0.8pt)/2,
    yshift=-6pt,
    draw,
    line width=0.8pt,
    font=\sffamily,
    rectangle,
    rounded corners,
    fill=blue!20,
    anchor=north west,
    inner sep=0.3333em,
    text width={\daywidth/#2-1.2em-1.6pt},
    minimum height=(#1*\hourheight)-2pt,
    align=center
}}

% Start the picture and set the x coordinate to correspond to days and the y
% coordinate to correspond to hours (y should point downwards).
\begin{tikzpicture}[y=-\hourheight,x=\daywidth]

    % First print a list of times.
    \foreach \time in {0,...,24}
        \node[anchor=north east] at (1,\time) {\time:00};

    % Draw some day dividers.
    \draw (1,0) -- (1,25);
    \draw (2,0) -- (2,25);
    \draw (3,0) -- (3,25);
    \draw (4,0) -- (4,25);
    \draw (5,0) -- (5,25);

    \node[anchor=north] at (1.5,-1) {調子いい!};
    % Write the entries. Note that the x coordinate is 1 (for Monday) plus an
    % appropriate amount of shifting. The y coordinate is simply the starting
    % time.
    \node[entry={8}{1}] at (1,1) {睡眠};
    \node[entry={1}{1}] at (1,9) {ご飯や身だしなみ};
    \node[entry={3}{1}] at (1,10) {朝数学};
    \node[entry={1}{1}] at (1,13) {ご飯など};
    \node[entry={4}{1}] at (1,14) {課題、英語、数学\\(前日に内容を決める)};
    \node[entry={0.5}{1}] at (1,18) {\scriptsize 振り返り。やりのこし列挙};
    \node[entry={1.4}{1}] at (1,18.65) {ご飯};
    \node[entry={3}{1}] at (1,20) {やりのこし、無ければサークルかバイト};
    \node[entry={1}{1}] at (1,23) {風呂、歯磨きは日付越えるまでに};

    \node[anchor=north] at (2.5,-1) {調子悪い!};
    \node[entry={3.5}{3}] at (2,9) {Class A};
    \node[entry={2.5}{3}] at (2.33333,9.5) {Class B};
    \node[entry={2.5}{3}] at (2.66667,10) {Class C};

    \node[anchor=north] at (3.5,-1) {昨日夜更かし今日急がし};

    \node[anchor=north] at (4.5,-1) {昨日夜更かし今日暇};
\end{tikzpicture}
\end{document}
