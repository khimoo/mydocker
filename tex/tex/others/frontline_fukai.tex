\documentclass[a4paper,12pt]{ltjsarticle}
\usepackage{amsmath,amssymb}

\begin{document}

\title{経済学の最前線期末レポート}
\author{日比野文\\202110833}
\date{\today}

\maketitle

\section*{問題1}
全ての年代を合計した出生率と世代毎のそれぞれの出生率について、相関関係が強いものを探す。\\
十代での出生率はほぼ変わらない。30歳以降での出生率は、ほぼ相関は無いように見えるが、強いて言えば負の相関である。
最も相関が強いのは20代での出生率である。20代での出生率が低いと全体の出生率も低くなる傾向がある。\\
このことから、女性が出産するペースは安定しており、第一子を産む年齢が遅くなっていることで、全体の出生率が低下していると考えられる。
\section*{問題2}
第5回講義資料より、子供の数nは$n=\frac{\delta}{1+\delta}\cdot\frac{1-\gamma}{\phi-\frac{p}{w}\theta}$で表される。大学にかかる教育コストを$\alpha$とすると、$n=\frac{\delta}{1+\delta}\cdot\frac{1-\gamma}{\phi-\frac{p-\alpha}{w}\theta}$が、国が大学の学費を負担した際の子供の数である。以下がなりたつ。
\begin{align*}
    p &> p-\alpha\ \ \ \ \ \ \because p>\alpha>0\\
    \phi-\frac{p}{w}\theta &< \phi-\frac{p-\alpha}{w}\theta\ \ \ \ \ \ \because \phi,w,\theta>0
\end{align*}
$\delta>0$なので、$1-\gamma>0 i.e. \gamma<1$のとき
$$
\frac{\delta}{1+\delta}\cdot\frac{1-\gamma}{\phi-\frac{p}{w}\theta}>\frac{\delta}{1+\delta}\cdot\frac{1-\gamma}{\phi-\frac{p-\alpha}{w}\theta}
$$
$1-\gamma<0 i.e. \gamma>1$のとき
$$
\frac{\delta}{1+\delta}\cdot\frac{1-\gamma}{\phi-\frac{p}{w}\theta}<\frac{\delta}{1+\delta}\cdot\frac{1-\gamma}{\phi-\frac{p-\alpha}{w}\theta}
$$
よって、国が大学の学費を負担すべきか否かは、$\gamma$に依存する。$\gamma$とは、子供の能力と親の投資の和の指数であり、子供の資産価値を決定する値である。通常能力が高く投資が大きいほど資産価値は上がるので、大抵は$\gamma>1$である。このことから大学授業料無償化は大抵の場合少子化対策に有効だと考えられる。

\section*{問題3}
大学無償化と出生率の増加に因果関係があることを示すためには、少なくとも何かしらの方法で大学の学費の免除を行いつつ無償化されなかった場合と無償化された場合の出生率の変化を比較する必要がある。\\
制度を実施しながらその効果を検証ために、子供のいる世帯を世帯年収、地域、職種などで分類し、それぞれの属性に向けた学費免除のための条件を実施し、
その後の出生率の変化を調べる。\\
例えば、世帯年収による免除、本籍地と大学が離れている場合の免除、公務員の場合の免除などを実施し、それぞれの条件に当てはまる世帯の出生率の変化を調べる。\\
このようにすることで、少なくとも教育コストと複合的な条件による出生率の上限があるか否かをある程度調べることができる。また、様々な免除を行った領域で出生率の増加が見られれば、因果関係があったという可能性が高まる。

別の方法として、年金制度のように、任意で18歳までのつみたて貯金や投資を行える制度を作る。\\
この方法では、制度開始前に子供を持つ世帯にあまりメリットがないため、制度開始前後の出生率の変化は、より学費無償化と因果関係が深いと考えられる。
\end{document}
