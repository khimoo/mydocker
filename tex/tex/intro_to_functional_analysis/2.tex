\documentclass[12pt,a4paper]{ltjsarticle}

    \usepackage{amsthm,amsmath,amssymb}

    \theoremstyle{definition}
    \newtheorem{dfn}{Definition}[section]
    \newtheorem{prop}[dfn]{Proposition}
    \newtheorem{lem}[dfn]{Lemma}
    \newtheorem{thm}[dfn]{Theorem}
    \newtheorem*{thm*}{Theorem}
    \newtheorem{cor}[dfn]{Corollary}
    \newtheorem{rem}[dfn]{Remark}
    \newtheorem*{rem*}{Remark}
    \newtheorem{ex}[dfn]{Example}
    \newtheorem{fact}[dfn]{Fact}
    \renewcommand{\qedsymbol}{$\blacksquare$}

%参考:https://qiita.com/hungry_and_fool/items/b28f9bb32ce738889523
%
%入力例:
%\section{二等辺三角形}
%\begin{dfn}
%二等辺三角形とは、2つの辺の長さが等しい三角形のことである。
%\end{dfn}
%\begin{prop}
%二等辺三角形の2つの底角の大きさは等しい。
%\end{prop}
%
%\section{正三角形}
%\begin{dfn}
%正三角形とは、すべての辺の長さが等しい三角形のことである。
%\end{dfn}
%\begin{prop}[正三角形と二等辺三角形]
%正三角形は二等辺三角形である。
%\end{prop}
%\begin{rem*}
%逆は成り立たない。すなわち、二等辺三角形がすべて正三角形とは限らない。
%\end{rem*}
%\begin{thm}
%\label{angle_eq}
%正三角形において、すべての角の大きさは等しい。
%\end{thm}
%\begin{proof}
%正三角形$ABC$は、$AB=AC$の二等辺三角形だから、2つの底角は等しい。すなわち$\angle ABC = \angle ACB$である。
%同様にして、$\angle BCA = \angle BAC, \angle CAB = \angle CBA$が成り立つ。
%したがって、$\angle ABC = \angle BCA = \angle CAB$が成り立つ。
%\end{proof}
%\begin{cor}
%正三角形の一つの内角の大きさは\ang{60}である。
%\end{cor}
%\begin{proof}
%Theorem \ref{angle_eq}と、三角形の内角の和が\ang{180}であることから従う。
%\end{proof}


\begin{document}

\title{関数解析入門第2回}
\author{日比野 文}
\maketitle

\section{Fourier級数が成立する別の例}
\begin{thm}
	$f\in C^2_{\mathrm{per}}$のときのFourier級数は収束
\end{thm}
\begin{proof}
	概略:Abelの総和法を用いる。
\end{proof}
\begin{rem}
	最初からフーリエ級数を表示するのではなく収束因子をかけて$r\to 1$とする
\end{rem}

\section{Poissonの定理}
\begin{thm}[Poissonの定理]
	$\forall f \in C_{\mathrm{per}}$に対して、$\mathrm{lim}_{\epsilon \to 0} Prf(x)=f(x)$が一様収束の意味で成立する。
\end{thm}
\begin{proof}
	概略:du Bois-Reymondの例と関数の同定、$\forall f \in C^2_{\mathrm{per}}$のときのFourier級数の収束が絶対収束かつ一様収束で成立すること、加えて3つの補題を用いる。
\end{proof}

Dirichlet核について2つの性質も紹介された。
\end{document}
