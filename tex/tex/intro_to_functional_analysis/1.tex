\documentclass[a4paper,10pt]{ltjsarticle}
\begin{document}

\title{関数解析入門第1回}
\author{日比野 文}
\maketitle

Fourier係数がどのように決まるのか2つの仮定のもと考察した。
その後仮定の妥当性について学んだ。
仮定2についてはどのような関数が仮定を満たすのかの確認が難しかったので
$C^m_{\mathrm{per}}$という集合を導入し、$m=2$のときの関数について確認した。
最後にTaylor展開の復習、比較を行い、Fourier級数の収束が必ずしもうまく行かない結果
が多くあるという例を学んだ。
test

\end{document}
